\section{Oefenopgaves}
\label{/tussentoets-ii/oefenopgaves}

1.  Als $y = 2 x + 0.6$ en de fout op $x$ is $\Delta x$, wat is dan de fout op $y$?\\


2. Als $y = -3 x + 2  x^2$ en de fout op $x$ is $\Delta x$, wat is dan de fout op $y$?\\


3. Als $y = \sin(x)$ en de fout op $x$ is $\Delta x$, wat is dan de fout op $y$?\\


4. Als $y = \frac{1}{x^2}$ en de fout op $x$ is $\Delta x$, wat is dan de fout op $y$?\\


5. Als $z = \frac{4 \cdot x}{3\cdot y}$ en de onzekerheden op $x$ en $y$ volgende de Poisson statistiek. Wat is dan de onzekerheid op $z$? Schrijf de onzekerheid zo kort mogelijk op.\\


6. Als $E = mc^2$ en de fouten op ($m$,$c$) zijn ($\Delta m$,$\Delta c$), wat is dan de fout op $E$?\\


7. Stel je wil de kinetische energie berekenen van een object. De formule is $E_k = 1/2 m \cdot v^2$. Je meet de massa van het object, deze is $m=2.3 \pm 0.2$ kg. De snelheid is $v=0.20 \pm 0.05$ m/s.\\
Bereken de kinetische energie.\\


8. We hebben een dataset met de gemeten massa van 80 kogels. Het gemiddelde van de massa-distributie is bepaald op 108.2 kg. De standaardafwijking van de massa-distributie is 11.2 kg. Wat is de fout op het berekende gemiddelde?\\


9. We combineren onafhankelijke datasets waarbij de spanwijdte van koolmeesjes zijn gemeten. Dataset A heeft informatie over 1100 koolmeesjes met een gemiddelde spanwijdte van 13.4 cm met een standaardafwijking van 2.0 cm. Dataset B heeft informatie over 2000 koolmeesjes met gemiddelde van 14.0 cm en een standaardafwijking van 1.8 cm.\\
Wat is het gemiddelde van de gecombineerde dataset T?\\


10. Onder welke voorwaarde mogen we aannemen dat de onzekerheid op het berekende gemiddelde van een dataset kleiner wordt als we datapunten toevoegen?\\


11. We hebben een dataset met metingen van een grootheid $x$ met precies 16 punten. Het gemiddelde waarde van $\overline{x} = 22$ met een standaardafwijking van $x = 4 $.
\begin{itemize}
\item[a] Wat is de onzekerheid op het gemiddelde van deze dataset?
\item[b] We voegen nog 9 extra waardes aan onze dataset toe. Wat is de onzekerheid op het gemiddelde nu?
\end{itemize}
\\

12. Je hebt de volgende dataset met waardes van x en y:\\
\begin{center} {2,5}, {1,4}, {5,2}, {3,0}\end{center}
\begin{itemize}
\item[a] Bereken de covariantie.
\item[b] Bereken de correlatie.
\end{itemize}
\\

13. Je hebt de volgende dataset met waardes van x en y:\\
\begin{center}{1,0}, {3,4}, {2,6}, {4,8}\end{center}
\begin{itemize}
 \item[a] Bereken de covariantie.
 \item[b] Bereken de correlatie.
\end{itemize}
   

14. Op een zomerse avond zie je rook en waar rook is is vuur. Op een zomerse avond is de kans dat je rook ziet (10\%), meestal door gebruik van barbeques. Gevaarlijke branden zijn heel zeldzaam (1\%) de kans dat rook bij een gevaarlijke brand vrijkomt is groot (90\%). Wat is de kans dat de rook die je ziet van een een gevaarlijk brand afkomt?\\


15. Van een dataset is het populatiegemiddelde en het steekproefgemiddelde bekend. $\mu = 5.3 $ kg en $s = 5.1$ kg. Bereken de onzuiverheid van de meeting.\\
 

16. Je meet met een meetlat die 2 cm te kort blijkt te zijn. Wat is de onzuiverheid van de metingen die je verricht?\\


17. Je weegt de massa van stenen op. De onzuiverheid van de weegschaal is bekend, deze is +42 gram. Je meet een massa van 160 gram.
\begin{itemize}
\item[a]  Wat is de daadwerkelijke massa van de steen?
\item[b] Mag je zo'n correctie toepassen?
\end{itemize}


\section{Antwoorden}
\label{/tussentoets-ii/antwoorden}

1. ${\Delta y = \sqrt{\left(\frac{\partial y}{\partial x}\right)^2 (\Delta x)^2 } = \sqrt{4 (\Delta x)^2}=2\Delta x}$\\


2. ${ \Delta y = \sqrt{\left(\frac{\partial y}{\partial x}\right)^2 (\Delta x)^2 } = \sqrt{(-3+4x)^2 (\Delta x)^2}= |{-3+4x}| \cdot \Delta x}.$\\


3. ${ \Delta y = \sqrt{\left(\frac{\partial y}{\partial x}\right)^2 (\Delta x)^2 } = \sqrt{(\cos x)^2 (\Delta x)^2}= \cos x \cdot \Delta x}.$\\


4. ${ \Delta y = \sqrt{\left(\frac{\partial y}{\partial x}\right)^2 (\Delta x)^2 } = \sqrt{\left( -\frac{2}{x^3} \right)^2 (\Delta x)^2}= \frac{2}{x^3} \cdot \Delta x}.$\\


5. We schrijven eerst de partiële afgeleides op van $z$ naar $x$ en $y$:\\
$\frac{\partial z}{\partial x} = \frac{4}{3y} = \frac{1}{x}z$ en \\
$\frac{\partial z}{\partial y} = - \frac{4x}{3y^2} = - \frac{1}{y} z.$\\
Vervolgens realiseren we ons dat $\Delta x = \sqrt{x}$ en $\Delta y = \sqrt{y}.$ Immers er staat dat $x$ en $y$ de Poisson statistiek volgen.\\
Nu vullen we dit in in de algemene vergelijking:\\
$\begin{aligned}
\Delta z & = \sqrt{ \left( \frac{\partial z}{\partial x}\right)^2 \left( \Delta x \right)^2 + \left( \frac{\partial z}{\partial y}\right)^2 \left( \Delta y \right)^2}\\
& =\sqrt{ \left( \frac{1}{x}z \right)^2  \left( \sqrt{x} \right)^2 +\left( - \frac{1}{y} z \right)^2 \left( \sqrt{y} \right)^2} \\
& = \sqrt{  \frac{x}{x^2} \cdot z^2 + \frac{y}{y^2} \cdot z^2} \\
& = z \cdot \sqrt{ \frac{1}{x} + \frac{1}{y}}
\end{aligned}$


6.  $ \Delta E = \sqrt{\left(\frac{\partial E}{\partial m}\right)^2 (\Delta m)^2 + \left(\frac{\partial E}{\partial c}\right)^2 (\Delta c)^2 } = \sqrt{(c)^2(\Delta m)^2 + (2mc)^2 (\Delta c)^2}.$\\


7.  ${E_k = 1/2 m \cdot v^2 = 0.046}$ J.\\
$\Delta E_k = \sqrt{\frac{\partial E_k}{\partial m})^2 ( \Delta m)^2 + \left(\frac{\partial E_k}{\partial v}\right)^2 \left( \Delta v \right)^2} = \sqrt{\left( \frac{1}{2}v^2 \Delta m\right)^2 + \left( mv \Delta m \right)^2 } = 0.02$ J\\
$E_k =  0.046 \pm 0.02$ J.\\


8. ${ \Delta \overline{m} = \frac{s}{\sqrt{n}} = \frac{11.2}{\sqrt{80}}} \text{ kg} = 1.25\text{ kg}$.\\


9. ${ \mu_A = \sum_a^{N_A=1100} \frac{x_a}{N_A} \text{ en } \mu_B = \sum_b^{N_B=2000} \frac{x_b}{N_B} \text{ ook geldt: }  \mu_T = \frac{\sum^{N_A} x_a + \sum^{N_B} x_b}{N_T}}$.\\
${ \mu_T =  \frac{N_A \cdot \mu_A + N_B \cdot \mu_B}{N_A + N_B}} = 13.8 \text{ cm}$.\\


10. Dat mogen we doen als voor de onderliggende verdeling van de dataset een goed gedefinieerde (eindige) variantie bestaat.\\


11.a De onzekerheid is: $s_{\overline{x}} (=\Delta \overline{x})= \frac{s}{\sqrt{n}} = 4/\sqrt{16} = 1 $.\\
11.b De onzekerheid is: $s_{\overline{x}} (=\Delta \overline{x})= \frac{s}{\sqrt{n}} = 4/\sqrt{25} = 0.8 $.\\


12.a $\bar{x} = (2+1+5+3)/4 = 2.75$, $\bar{y} = (5+4+2+0)/4 = 2.75$\\
$\text{cov}(x,y) = \frac{1}{n} \sum_n (x_i- \bar{x})\cdot (y_i - \bar{y})$\\
$ = \frac{1}{4}\cdot ((2-2.75)\cdot(5-2.75) +(1-2.75)\cdot(4-2.75)+(5-2.75)\cdot(2-2.75) +(3-2.75)\cdot(0-2.75))$\\
$ = -1.6$.\\
12.b $s_x^2 = \overline{~x^2}-\bar{x}^2 = (4+1+25+9)/4 - 2.75^2 = 2.1875$,\\
$s_y^2 = \overline{~y^2}- \bar{y}^2 = (25+16+4+0)/4 - 2.75^2 = 3.6875$,\\
$s_x = 1.48$, $s_y = 1.48$,\\
$r_{xy} = \frac{\text{cov}_{xy}}{s_x s_y} = \frac{-1.56}{1.48\times 1.92} = -0.55$.\\



13.a $\bar{x} = 2.5$, $\bar{y} = 4.5$,\\
$\text{cov}(x,y) = \frac{1}{n} \sum_n (x_i-\bar{x})\cdot (y_i - \bar{y}) = 2.75$\\
13.b $s_x^2 = \overline{~x^2}-\bar{x}^2 = 1.118^2$,\\
$s_y^2 = \overline{~y^2}-\overline{y}^2 = 2.958^2$,\\
$r_{xy} = \frac{\text{cov}_{xy}}{s_x s_y} = \frac{2.75}{1.118\times 2.958} = 0.83$.\\


14. Gebruik Bayes theorema om dit uit te rekenen:\\
$\begin{aligned}
    \displaystyle 
      P(\text{gevaarlijke brand} \mid \text{rook}) 
      & = \frac{P(\text{rook} \mid \text{gevaarlijke brand})\cdot P(\text{gevaarlijke brand})}{P(\text{rook})} \\
      & = \frac{0.9 \times 0.01}{0.10} = 0.09.
\end{aligned}$


15. De onzuiverheid is: $b = s - \mu = 5.1 - 5.3$ kg $= -0.2$ kg. \\


16. De onzuiverheid  = - 2 cm


17.a De eigenlijke massa van de steen is 160 - 42 = 128 gram.\\
17.b Ja, zo'n correctie mag je wel toepassen.

